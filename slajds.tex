\documentclass{beamer}
\usepackage[utf8]{inputenc}
\usepackage[swedish]{babel}


\title{A Naive Sudoku Solver\\\small{PKD group 1}}
\author{Alexander Andersson \and Alvar Bjerkeng van Keppel}
\date{\today}

% slajd 0
\begin{document}
\begin{frame}[introslide]
  \titlepage
\end{frame}

% slajd 1
\begin{frame}
  \frametitle{Datastruktur och algoritm}
\begin{description}
\item[Datastruktur] $\texttt{abstype board}$ lagrar de möjligheter som finns för cellerna i brädet. Kastar $\texttt{NotASolution}$ om resultatet inte har m\"ojligheter i n\aa gon ruta.
  \begin{description}
  \item[$\texttt{emptyboard}$] Givet $n$,
    initiera ett tomt bräde av storlek $n\times n$.
  \item[$\texttt{setCell}$] Givet ett bräde $b$, $(x,y)$ och $n$,
  ge en uppdaterad version av $b$ där möjligheterna på
  position $(x,y)$ är $\{n\}$ och \emph{uppenbara} omöjligheter utesluts.
\item[$\texttt{getCell}$] Givet ett bräde $b$, $(x,y)$,
  ge möjligheterna på plats $(x,y)$ i $b$ som en lista.
  \end{description}
\item[Algoritm] Gissa succesivt på den första rutan som har minimalt
  antal möjligheter. Implementeras i $\texttt{findFirstSolution : board -> board option}$.
\end{description}
\end{frame}

% slajd 2
\begin{frame}
  \frametitle{Hjälpfunktioner och UIet}
  % nämn något intressant urval av de funktioner
  % som skyfflar data mellan olika datatyper
  % och IO.

  % visa hur UIet funkar
\end{frame}

% slajd 3
\begin{frame}
  \frametitle{Prestanda/Demo}
  % Om du vill visa upp hur kickass programmet är. typ.
\end{frame}

% slajd 4
\begin{frame}[t]
  \frametitle{Bjällror och visselpipor som inte blev implementerade}
\begin{description}
\item[Parallellisering] Måste kunna starta ett antal workers, skicka jobb till den minst belastades kö, returnera värden till huvudprocessen.
  \pause
  \begin{description}
  \item[Concurrent ML] Till SML/NJ. Fanns en del dokumentation, inga exempel inom domänen.
    \pause
  \item[PolyML] Ingen direkt dokumentation med genomgångar. Använde locking och waiting.
    \pause
  \item[Isabelle ML] Specifikt för matematiska bevis. Bygger på PolyML... bland andra.
  \end{description}
\end{description}
\end{frame}

%slajd 5
\begin{frame}
  \frametitle{Problem och slutsater}
  % Ja, vad tycker vi? allt tar mycket längre tid än vad man tror,
  % börja med rapporten tidigare?
  % Det är jobbigt att inte ha bra dokumentation till paralleljoxet?

  \begin{itemize}
  \item Tester är asbra, de hittar logiska fel och regressioner
  \item På ett tomt $25\times 25$ bräde upptäcker man att $n^4$ växer snabbt
  \item Rapportskrivande bör påbörjas i tid
  \item Parallelisering \"ar sv\aa rt utan bra dokumentation
  \item Ganska nöjda med slutresultatet, ett fungerande program med spartanskt UI.
  \end{itemize}
\end{frame}

\end{document}
