\documentclass[12pt, a4paper]{article}
\usepackage[utf8]{inputenc}
\usepackage[swedish]{babel}
\usepackage{fullpage, hyperref, amsmath, amssymb, mathtools}
\usepackage[parfill]{parskip}
\usepackage{color}

\title{A Naive Sudoku Solver\\\small{PKD group 1}}
\author{Alexander Andersson, 860616-1530 \and Alvar Bjerkeng van Keppel}
\date{\today}

\begin{document}

\maketitle

\begin{abstract}

\end{abstract}

\tableofcontents

\section{Idea}

% naiv backtracking
% parallelisering om det finns tid över
The project idea was to implement a sudoku solver in ML using backtracking and to add bells and whistles if time permits. On the list of bells and whistles are to make the solver search for different solutions in parallel and a nice user interface.

\section{Implementation}

\subsection{Introduction}

The sudoku is a popular brain teaser with the objective of filling a $9\times 9$ board with the numbers 1-9.

The Sudoku puzzle can be generalized slightly by allowing the side of the board to be the square of a positive integer (ie $1, 4, 9, 25, \dots$). This is the puzzle the implementation described in this report solves.

\subsection{$\texttt{abstype board}$}
The main data structure is the abstract data type $\texttt{board}$. $\texttt{board}$ is of type int * int list vector. The integer specifies the boxlength. The vector has one index for each cell in the board. There is a mapping from $(x,y)$ coordinates defined by $\texttt{indexToxy : int -> int * int}$ and the almost inverse $\texttt{xyToIndex : int -> int -> int}$ to and from the indexes in the vector. the list at index $\texttt{xyToIndex i j}$ represents all the numbers that have not been judged impossible at $(x,y)$. As of writing this, $k$ is judged impossible at $(i,j)$ if and only if there is a $(x,y)$ such that
\begin{itemize}
\item $(x,y)$ is distinct from $(i,j)$
\item $(x,y)$ lies in the same row, column or box as $(i,j)$
\item the list at $(x,y)$ is $[k]$.
\end{itemize}

\subsection{Constraint Propagation}

$\texttt{setCell}$ is the function that does the heavy lifting to enforce the constraints imposed by putting numbers in cells. $\texttt{setCell}$ takes a board, an $(x,y)$ coordinate and a number $k$ to place at the given position. It returns a new board with the $(x,y)$ index set to $[k]$ where $k$ is removed from the possible values of the cells in the same row, column or box. If one of these cells become a singleton, the same thing is done until nothing needs updating for the invariant to hold.

\subsection{Solving Algorithm}

As of this writing the solving algorithm tries to find one solution. This is done by the following steps
\begin{enumerate}
\item Take the current board if it is completely determined. Otherwise...
\item Find a cell with more than one possibility but with as few possibilities as possible
\item Successively guess at the possibilities for cell and
        apply the solving algorithm on the new board
\item If one is found, return it, otherwise fail
\end{enumerate}

The function implementing this algorithm is $\texttt{findFirstSolution : board -> board option}$ which returns $\texttt{SOME solution}$ if a solution is found and $\texttt{NONE}$ otherwise. The algorithm should find a solution if one exists, given enough time.

\subsection{Conversion Functions}


There are a few functions that takes or gives a board that are of interest.
\begin{description}
\item[$\texttt{printBoard : board -> string}$] Returns the numbers determined
  in the board separated by commas between cells and newlines
  between rows in board.
\item[$\texttt{boardToListList : board -> int list list list}$]
  Returns L where (List.nth (List.nth L y) x) is the list of numbers possible
  at $(x,y)$ in the board.
\item[$\texttt{readBoard : string list -> board}$]
  Takes a string representation of a partially filled board and returns a board.
\end{description}

\section{Correctness}

It seems to be working although no correctness proof have been made and no large amount of test. It is able to pass the (albeit few) testcases in $\texttt{main\_tests.sml}$.

\section{Shortcommings}

The user interface is a bit spartan and could be made more snuggly. The constraint propagation is very naive and could be improved. The debug function in the board abstype is still left for testing. This is suboptimal and could probably be dealt with more optimally. There is room for structuring the code better, at the time of writing, most of the functions are in the file $\texttt{main.sml}$.

\section{Usage Instructions}

The user interface is through the script $\texttt{sudokusolver}$. It is possible to invoke it in a few ways. The tool will print a solution if one exists or print a message if not.
\begin{itemize}
\item Pipe into the tool as in $\texttt{otherprogram | sudokusolver}$
\item Redirect a file into it as in $\texttt{sudokusolver < in.txt}$
\item Giving the script an argument as in $\texttt{sudokusolver in.txt}$
\end{itemize}
This is an example of an input file.
\begin{samepage}
\begin{verbatim}
8,6,2, , , , , ,
 , , , ,6, ,7,9,
 , , , , ,2, , ,
 , , ,5, , ,9, ,
 , , , , , , ,7,
 , , ,9, , ,3, ,
1,5, , , , , , ,
 , , , ,2, , , ,4
9, , , ,3, , ,6,
\end{verbatim}
  There should be $n-1$ rows and $n-1$ commas for each row.
  A number (surrounded with an arbitrary amount of whitespace before the $k$th comma represents that number on the $k$th position on that row. Just whitespace represents an empty cell. A newline represents the end of a row. There is a couple of examples in the folder $\text{puzzlefiles/}$.

\end{samepage}
\section{Conclusion}

For $9\times9$ sudokus the time to solve a board is below half a second, including the time to compile the whole project.

\end{document}
